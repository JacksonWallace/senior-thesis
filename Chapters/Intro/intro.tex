\chapter{Introduction}
\label{chap:intro}
The Standard Model (SM) is one of the crown jewels of 20$^{\text{th}}$ century physics. With a handful of parameters, the theory allows for the precise prediction of numerous experimental results in particle physics. After the discovery of the Higgs boson in July 2012 by the CMS and ATLAS collaborations~\cite{atlas2012, cms2012}, the model is complete. However, despite the successes of the SM, there are still several questions remaining in particle physics to which the SM does not have a complete answer. For instance, the SM is silent on the subject of neutrino masses~\cite{Capozzi_2016}. Recent experimental results have heightened the tension between the predicted and measured muon anomalous magnetic moment~\cite{PhysRevLett.126.141801}. One of the more intriguing mysteries with no forthcoming answer from the SM is that of dark matter (DM)~\cite{Planck:2020cp,Zwicky_1933}. A multitude of theories exist on the nature of DM and yet ceaseless experimental efforts have brought physicists no closer to understanding it. Further indirect detection efforts involving a single SM particle recoiling off another particle, which is read out by detectors as missing transverse momentum (\ptmiss), may yet provide answers behind the source of dark matter.

The principal source of evidence for the existence of dark matter comes from the rotation of galaxies~\cite{Rubin_1970}. The observed rotational velocity of galaxies as a function of the distance to their center does not agree with the velocity predicted by Newtonian mechanics, with the velocity remaining nearly flat instead of falling proportionally to the inverse square root of the radial distance. This discrepancy led to the suggestion that galaxies contain a halo of additional mass that seems to only interact gravitationally with other matter. There are numerous other pieces of evidence supporting this hypothesis, including the collisions of galaxy clusters like the Bullet Cluster~\cite{Thompson_2015} and even the Cosmic Microwave Background~\cite{Planck:2020cp}.

A variety of theoretical models have been proposed to explain the aforementioned phenomena~\cite{dm2005,Arcadi_2018}. Of particular interest to experiments are models wherein DM interacts with SM in some way, which is a necessity to detect it. One common avenue to search for dark matter is indirect searches that look not for dark matter itself, but for deviations from the SM that could be a sign of dark matter~\cite{Gaskins_2016}. In particle colliders, this often means looking at a single SM object such as a jet, Z boson, or Higgs boson recoiling off of another particle that the detector reports as \ptmiss~\cite{dms2018}. These final states are abbreviated as mono-X.

The Large Hadron Collider (LHC) accelerates protons at $\TeV{13}$ and collides them in several detectors built along the $\SI{26.7}{\km}$ tunnel that houses the collider~\cite{Evans_2008}. These detectors study the particles produced in these collisions to make precision measurements of the SM or to search for physics beyond the SM (BSM). One of these detectors is the Compact Muon Solenoid (CMS). CMS gets its name from the large $\SI{4}{\tesla}$ solenoid magnet that curves the paths of charged particles in the detector. CMS consists of a silicon tracker, an electromagnetic calorimeter (ECAL), a hadron calorimeter (HCAL), a system designed for detecting muons, and calorimeters close to the beam path~\cite{Collaboration_2008}. Events that are read out at CMS are selected by a system of hardware and software called the trigger. 

The Large Hadron Collider is being upgraded to allow for more protons to be collided per second~\cite{Collaboration:2650976}. The number of collisions per proton bunch interaction is referred to as the pileup (PU). This change will usher in the era of the High-Luminosity LHC (HL-LHC). To complement these changes, the CMS Phase-2 upgrade is designed to allow the detector to handle the increase in luminosity while also providing other upgrades that will increase the acceptance of a multitude of analyses. 

In order to handle the increased luminosity, the CMS triggers must be upgraded. The first-level trigger will now receive additional inputs from other detector systems. In order for these systems to be compatible with the new trigger, the electronics of the tracker, ECAL, HCAL, and muon systems will all be upgraded. These systems themselves will also receive substantial upgrades including improved resistance to the increased radiation of the HL-LHC and heightened detector granularity in order to distinguish between the many more particle interactions that will be occurring. Finally, new systems will be added to the detector. The endcap detectors will be replaced with the HGCAL, while the Minimum Ionizing Particle Timing Detector (MTD) will allow the detector to gain further temporal resolution to help mitigate PU.
Further specifics on the Phase-2 upgrade will be summarized in \cref{section:phase2}. A full description can be found in~\cite{Collaboration:2650976, Contardo:2020886, CERN-LHCC-2020-004, CERN-LHCC-2017-011, CERN-LHCC-2017-023, CERN-LHCC-2017-012, CERN-LHCC-2017-009, CMS:2667167, Collaboration:2759072}.

The HL-LHC and the corresponding CMS Phase-2 upgrades will result in improvements of previous searches for dark matter. One of the primary benefits is the increased luminosity of the HL-LHC. The increase in the number of collisions will allow analyses to probe rare processes, including several mono-X final states. As described in \cref{section:2hdma}, uncertainties in the mono-Higgs final state are still dominated by statistical uncertainties and there is room for improvement on previous searches such as \cite{cms:hbb2019,atlas:hbb2017,atlas:hbb2021} by examining higher luminosities.

Another way in which searches for dark matter will be improved comes from the increased geometric acceptance of the detector. With parts of the detector extending to further pseudorapidities and the addition of the MTD, the extent of reconstructable physics objects will also increase. In the case of mono-X searches, this will allow for increased background reduction by lowering the rate of the detector misidentifying particles.

Finally, with the upgrades to CMS and the LHC, there will be corresponding upgrades in the tools and algorithms used in searches. During Run 2 of the LHC, there was an increase in usage of machine learning tools to effectively identify jets reconstructed from the hadronic decays of massive SM particles~\cite{CMS:2020mlt}. Further improvements to these tools, referred to as taggers, will enable increased sensitivity to BSM physics at the HL-LHC.

Given the still unknown nature of dark matter and the numerous advances that will be made with particle colliders in the next decade, it is imperative to understand the role that these colliders can play in helping to resolve one of physics' largest outlying mysteries. Accordingly, I have undertaken a projection study that aims to understand the future outlook for dark matter searches at CMS. As the width and depth of all dark matter searches would far exceed the scope of this study, I focus on searches for dark matter with a mono-Higgs + \ptmiss final state, specifically one where the Higgs has a high momentum and decays into a b quark-antiquark pair. I simulate a search for such a final state at the Phase-2 CMS detector. I then interpret the results through the lens of the 2HDM+a model, which has parameters that are sensitive to this final state. These results were initially presented at the Snowmass Energy Frontier Workshop in 2022~\cite{CMS-PAS-FTR-22-005}.
I find that the HL-LHC, with an expanded $\fbinv{3000}$ dataset, will have additional sensitivity to BSM physics. If observed, 2HDM+a signals with $m_\mathrm{a} = \GeV{250}$ and $m_\mathrm{A}$ from $1000$ to $\GeV{1600}$ could have a significance indicating discovery.
If no significant excess is observed, the exclusion of parameters of the 2HDM+a model can be extended beyond previous results. $m_\mathrm{A}$ values from $750$ to $\GeV{2000}$ could be excluded for $m_\mathrm{a}$ equal to $\GeV{250}$, while signals with $m_\mathrm{A}$ from $1250$ to $\GeV{1600}$ could be excluded for $m_\mathrm{a} = \GeV{500}$. Because of the additional sensitivity that the HL-LHC will have, it is well worth continuing to probe this final state in search of new physics.
